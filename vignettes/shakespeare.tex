\documentclass{article}
\usepackage{extR}
\usepackage[numbers]{natbib}

\title{ Interfaces from R and Extensions \\ (or, Shakespeare
  Meets R)}

\author{John M Chambers}
\date{\today}

\begin{document}

\maketitle
\bibliographystyle{plainnat}

\section{Introduction}
\label{sec:introduction}

In the modern software-rich environment, a project with serious
computing needs is likely to benefit from using tools in more than one
programming language, as long as the interface allows users easy
access to the computing they need without a serious learning barrier.

In the following sections, some techniques are explored that may help
implementers provide such techniques for application users.
We will build on a general approach to interfaces described in
\cite{extR}.
The running example will be a package with a modest goal, the use of
some data analytic techniques to explore Shakespeare's plays for
possibly interesting patterns.
Admittedly not of great importance compared to the major modern
challenges, but a subject that has generated much discussion for
centuries.
Facilities for Interfaces and object-oriented programming in \R{} will
benefit the application, as is often the case.  We will also emphasize programming to extend
these techniques.

\subsection{Interfaces and Extensions}\label{interfaces-and-extensions}

Using an interface from one programming language or environment to
another is a winning strategy for projects with challenging computing
requirements. In the ``Extending R'' book (@extR, hereafter \emph{ExR})
this strategy is related to R, as one of the three fundamental
principles: ``Interfaces are part of R''.

When essential programming idioms are shared between the languages, the
use of an interface can be made natural for programming by defining
\emph{proxies}. In particular, both R and many other powerful languages
emphasize versions of \emph{functions} and \emph{classes}.

An interface from R to such a language can define proxy functions and
proxy classes, to be used in a natural R way but which in fact are
automatically interfaced to corresponding program constructs in the
other language. In addition to using such proxies straight-out, the
application is likely to extend the computations and incorporate them in
functions, classes or other structures in R.

We'll look at some techniques for such extensions, particularly for
embedding proxy classes in R classes. For a more-or-less realistic
project in which these might be useful, we'll consider a package to
apply data analysis to the plays of William Shakespeare.

\subsection{Shakespeare Meets R, and XML, and
Python}\label{shakespeare-meets-r-and-xml-and-python}

For over four centuries, Shakespeare's plays have been the focus of a
sea of studies and debates, including some recent work applying
statistical and natural-language techniques, focussed on questions of
authorship in particular.

In a more exploratory spirit, there are a number of interesting
questions for which R is suited (well, interesting to me, certainly).
How do the plays change with time, or with subject matter? Are
characters distinguished in speech by class, by gender or by other
context?

Suppose we start on a project to construct tools in R to explore such
questions. First question: the data. The plays have been digitized, in
some different forms. For our purposes, one of these has the key
advantage of preserving the most structural information in a way that
can be useful for analysis. Thirty-seven plays were transcribed in HTML
format and later converted to XML.

The great asset of XML is that it provides a hierarchical grammar for
describing objects in a tree-like structure, using specific terms for
the particular objects. For the plays, this organizes the data by acts
and scenes. Within each scene speeches, stage directions and a few other
items are explicitly distinguished. All this structure is available to
us for analysis.

For example, the most important part of the plays will be the speeches,
those ``words, words, words'' as Hamlet says. In XML parlance, a speech
has a corresponding tag and structure:

\begin{verbatim}
<SPEECH>
<SPEAKER>OBERON</SPEAKER>
<LINE>Ill met by moonlight, proud Titania.</LINE>
</SPEECH>
\end{verbatim}

\section*{References}
\bibliography{shakespeare}

\end{document}
