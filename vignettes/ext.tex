\documentclass{article}
\usepackage{extR}

\bibliographystyle{plainnat}

\begin{document}
This vignette considers techniques for extending proxy classes in R
that map to classes in another language, to make computing facilities
in that language available in an R package. The proxy classes can be
used as fields or superclasses for R classes that extend the
capabilities from the interface. We'll look at some alternative
approaches and the programming to implement them.

\subsection{Interfaces and
  Extensions}\label{interfaces-and-extensions}

Using an interface from one programming language or environment to
another is a winning strategy for projects with challenging computing
requirements. In the \dQuote{Extending R} book \cite[hereafter
\textbf{XR}]{extR} this strategy is related to R, as one of the three
fundamental principles: \dQuote{Interfaces are part of R}.

When essential programming idioms are shared between the languages,
the use of an interface can be made natural for programming by
defining \emph{proxies}. In particular, both R and many other powerful
languages emphasize versions of \emph{functions} and \emph{classes}.

An interface from R to such a language can define proxy functions and
proxy classes, to be used in a natural R way but which in fact are
automatically interfaced to corresponding program constructs in the
other language. In addition to using such proxies straight-out, the
application is likely to extend the computations and incorporate them
in functions, classes or other structures in R.

We'll look at some techniques for such extensions, particularly for
embedding proxy classes in R classes. For a more-or-less realistic
project in which these might be useful, we'll consider a package to
apply data analysis to the plays of William Shakespeare.

\subsection{Shakespeare Meets R, and XML, and
  Python}\label{shakespeare-meets-r-and-xml-and-python}

For over four centuries, Shakespeare's plays have been the focus of a
sea of studies and debates, including some recent work applying
statistical and natural-language techniques, focussed on questions of
authorship in particular.

In a more exploratory spirit, there are a number of interesting
questions for which R is suited (well, interesting to me, certainly).
How do the plays change with time, or with subject matter? Are
characters distinguished in speech by class, by gender or by other
context?

Suppose we start on a project to construct tools in R to explore such
questions. First question: the data. The plays have been digitized, in
some different forms. For our purposes, one of these has the key
advantage of preserving the most structural information in a way that
can be useful for analysis. Thirty-seven plays were transcribed in
HTML format and later converted to XML.

The great asset of XML is that it provides a hierarchical grammar for
describing objects in a tree-like structure, using specific terms for
the particular application.
Data description can be complete and specialized, with the potential
to validate that a specific file  conforms to the requirements.

For the plays, this organizes the data by acts
and scenes within acts. Within each scene speeches, stage directions and a few
other items are explicitly distinguished. All this structure is
available to us for analysis.

For example, the most important part of the plays will be the
speeches.  In XML parlance, a speech has a corresponding tag and
structure:

\begin{verbatim}
<SPEECH>
<SPEAKER>OBERON</SPEAKER>
<LINE>Ill met by moonlight, proud Titania.</LINE>
</SPEECH>
\end{verbatim}

All the text in the speech is available, separated by lines (only one
here); in addition, the speaker is explicitly identified. From further
up the tree, we know the scene, the act and the play.

What tools might be useful to analyse such data? Again, it's the
speeches that really \emph{are} the plays, those \dQuote{words, words,
words} as Hamlet says. Analyzing the words can benefit from
natural-language tools that break down the text into \dQuote{tokens} and
attempt to build structure on these. A particularly popular and widely
used collection of such tools is \texttt{NLTK}, the Natural Language
Toolkit implemented in Python.

So, this is starting to look like a strategy. Obtain the data in the
form of XML files, make use of \texttt{NLTK} to ask questions about
the speeches and apply R to explore the results. An approach to
interfacing the three languages is needed next.

XML excels at representing structure, but it is very
hierarchical. Both R and Python are happier with \dQuote{linear}
structures.
Computations in R build on vectors and vector
structures such as matrices and data frames.
In Python, iterators and iterable
structures, such as Python lists are central.
Both languages will find it more natural to work with the speeches
through classes of objects that spread the hierarchical form into a
list whose elements are objects for each speech.

Both R and Python can parse XML files into corresponding internal
forms (in several ways in R). The tools of the \texttt{NLTK} will
usually be the first step in processing, implying that converting the
XML to Python is the obvious approach.

The specific strategy chosen is to parse the files in Python, which
produces objects of a particular class (\texttt{"ElementTree"}). From
these, we will construct other objects, most importantly lists of
speeches. Specialized Python classes will describe the objects.

R's main role is to supply the range of analysis and visualization
tools. In addition, two features of R set the basic approach to the
project: R packages and the R session. The package is the natural way
to organize a project at this scale. The software we'll look at will
be part of the \textbf{shakespeare} package, including R and Python
software as well as the files for the original data and optionally for
intermediate data forms as well.

R's package structure allows for essentially arbitrary files of source
for whatever languages are used. In our project, one folder contains
the XML files for the plays and another the Python source specially
written for this package.

The R session provides a continuing computing environment, interacting
with packages loaded into the session. We'll use that to organize the
data additionally to make accessing the plays and iterating over them
quick and simple. The files of XML data and the interface to Python,
for example, allow us to compute and cache data for the plays
providing rapid access throughout the session.

\subsection{The XR Approach: Proxy Functions and
  Classes}\label{the-xr-approach-proxy-functions-and-classes}

The goals of the \textbf{XR} approach to interfaces include generality
in the computations supported and simplicity in the user
interface. Both goals benefit from an approach to proxies; that is,
objects in R automatically created as proxies (references) to
analogous objects in the server language, here Python. Computations in
the server return proxy objects to R.

The R package includes additional proxies for functions and classes in
the server language. Users deal with the functions as if they were R
functions, supplying arguments that may transparently be either
ordinary R objects (which will be converted to server language
analogues) or proxies from previous computations.

\section*{References}

\bibliography{shakespeare}
\end{document}